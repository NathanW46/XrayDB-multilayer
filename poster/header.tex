%%
%%  Periodic Table of X-ray absorption edges and emission lines
%%
%%  using data from Elam, Ravel, and Sieber,
%%  Radiation Physics and Chemistry 63 (2002)
%%
%%  latex sources based on periodic_table  from Ivan Griffin,
%%
%%  modified by Matt Newville, Jan-2013
%%
%%
%% This work may be distributed and/or modified under the
%% conditions of the LaTeX Project Public License, either version 1.3
%% of this license or (at your option) any later version.
%% The latest version of this license is in
%%   http://www.latex-project.org/lppl.txt
%% and version 1.3 or later is part of all distributions of LaTeX
%% version 2005/12/01 or later.
%%
%%%%%%%%%%%%%%%%%%%%%%%%%%%%%%%%%%%%%%%%%%%%%%%%%
\documentclass[]{article}

\usepackage{verbatim}
\usepackage{tikz}
\usepackage[active,tightpage]{preview}

\usetikzlibrary{shapes,calc,arrows}


  \pgfdeclareimage[height=35mm]{Curie}{images/Marie_Curie}
  \pgfdeclareimage[height=34mm]{Barkla}{images/Charles_Barkla}
  \pgfdeclareimage[height=34mm]{Moseley}{images/Henry_Moseley}
  \pgfdeclareimage[height=34mm]{Mendeleev}{images/Dmitri_Mendeleev}
  \pgfdeclareimage[width=20mm]{qrxas}{images/xraytable_qr}
  \pgfdeclareimage[width=20mm]{qrgse}{images/gsecars_qr}

 \newcommand{\Moseley}{ \pgfbox[center,bottom]{\pgfuseimage{Moseley}}\\   Henry Moseley \\}
 \newcommand{\Curie}{\pgfbox[center,bottom]{\pgfuseimage{Curie}}\\   Marie Sklodowska Curie\\ }
 \newcommand{\Barkla}{\pgfbox[center,bottom]{\pgfuseimage{Barkla}}\\   Charles G. Barkla\\}
 \newcommand{\Mendeleev}{\pgfbox[center,bottom]{\pgfuseimage{Mendeleev}}\\      Dmitri Mendeleev \\}


%% Fill Color Styles
  \tikzstyle{ElementFill} = [fill=yellow!6]
  \tikzstyle{Element} = [draw=black, ElementFill,
  minimum width=70mm, minimum height=70mm, node distance=70mm]

  \tikzstyle{SpaceFill} = [fill=white]
  \tikzstyle{Space} = [draw=white, SpaceFill,
  minimum width=5mm, minimum height=5mm, node distance=5mm]

  \tikzstyle{TitleLabel} = [font={\sffamily\Huge\bfseries}, scale=3.50]
  \tikzstyle{SubTitleLabel} = [font={\Huge}, scale=2.50]

  \definecolor{MedRed}{rgb}{0.8,0,0}
  \definecolor{MedBlue}{rgb}{0,0,0.8}
  \definecolor{DeepBlue}{rgb}{0,0,0.6}


  \newcommand{\Color}[2]{{\textcolor{#1}{#2}}}
  \newcommand{\BRed}[1]{{\Color{MedRed}{\textbf{#1}}}}
  \newcommand{\BBlue}[1]{{\Color{MedBlue}{\textbf{#1}}}}
  \newcommand{\Red}[1]{{\Color{MedRed}{#1}}}
  \newcommand{\Blue}[1]{{\Color{MedBlue}{#1}}}
  \newcommand{\Name}[1]{{\Color{DeepBlue}{\textsf{\textbf{#1}}}}}
